% DO NOT MODIFY THIS FILE DIRECTLY.  IT IS CREATED BY mf6ivar.py 

\item \textbf{Block: OPTIONS}

\begin{description}
\item \texttt{BOUNDNAMES}---keyword to indicate that boundary names may be provided with the list of release-point cells.

\item \texttt{stoptime}---real value defining the maximum simulation time to which particles in the model can be tracked.  Particles that have not terminated earlier due to another termination condition will terminate when simulation time STOPTIME is reached.  If the last stress period in the simulation consists of more than one time step, particles will not be tracked past the ending time of the last stress period, regardless of STOPTIME.  If the last stress period in the simulation consists of a single time step, it is assumed to be a steady-state stress period, and its ending time will not limit the simulation time to which particles can be tracked.

\item \texttt{stoptraveltime}---real value defining the maximum travel time over which particles in the model can be tracked.  Particles that have not terminated earlier due to another termination condition will terminate when their travel time reaches STOPTRAVELTIME.  If the last stress period in the simulation consists of more than one time step, particles will not be tracked past the ending time of the last stress period, regardless of STOPTRAVELTIME.  If the last stress period in the simulation consists of a single time step, it is assumed to be a steady-state stress period, and its ending time will not limit the travel time over which particles can be tracked.

\item \texttt{STOP\_AT\_WEAK\_SINK}---is a text keyword to indicate that a particle is to terminate when it enters a cell that is a weak sink.  By default, particles are allowed to pass though cells that are weak sinks.

\item \texttt{istopzone}---integer value defining the stop zone number.  If cells have been assigned IZONE values in the GRIDDATA block, a particle terminates if it enters a cell whose IZONE value matches ISTOPZONE.  An ISTOPZONE value of zero indicates that there is no stop zone.  The default value is zero.

\item \texttt{OUTPUT\_FOR\_INACTIVE}---is a text keyword to indicate that output is to be written for inactive particles.  By default, output is not written for inactive particles.

\item \texttt{DRAPE}---is a text keyword to indicate that a particle is to be moved to the topmost active cell prior to release if its release point is in a cell that is dry at the scheduled time of release.  By default, a particle does not get released into the simulation if its release point is in a cell that is dry at the scheduled time of release.  ??? Move to what elevation within topmost active cell ???

\end{description}
\item \textbf{Block: DIMENSIONS}

\begin{description}
\item \texttt{nreleasepts}---is the number of particle release points.

\end{description}
\item \textbf{Block: PACKAGEDATA}

\begin{description}
\item \texttt{irptno}---integer value that defines the PRP release point number associated with the specified PACKAGEDATA data on the line. IRPTNO must be greater than zero and less than or equal to NRELEASEPTS.  The program will terminate with an error if information for a PRP release point number is specified more than once. ??? DO WE REALLY NEED THIS ???

\item \texttt{cellid}---is the cell identifier, and depends on the type of grid that is used for the simulation.  For a structured grid that uses the DIS input file, CELLID is the layer, row, and column.   For a grid that uses the DISV input file, CELLID is the layer and CELL2D number.  If the model uses the unstructured discretization (DISU) input file, CELLID is the node number for the cell.

\item \texttt{xrpt}---real value that defines the x coordinate of the release point in model coordinates.  The (x, y, z) location specified for the release point must lie within the cell that corresponds to the specified cellid.

\item \texttt{yrpt}---real value that defines the y coordinate of the release point in model coordinates.  The (x, y, z) location specified for the release point must lie within the cell that corresponds to the specified cellid.

\item \texttt{zrpt}---real value that defines the z coordinate of the release point in model coordinates.  The (x, y, z) location specified for the release point must lie within the cell that corresponds to the specified cellid.

\item \texttt{boundname}---name of the release-point cell.  BOUNDNAME is an ASCII character variable that can contain as many as 40 characters.  If BOUNDNAME contains spaces in it, then the entire name must be enclosed within single quotes.

\end{description}
\item \textbf{Block: PERIOD}

\begin{description}
\item \texttt{iper}---integer value specifying the stress period number for which the data specified in the PERIOD block apply. IPER must be less than or equal to NPER in the TDIS Package and greater than zero. The IPER value assigned to a stress period block must be greater than the IPER value assigned for the previous PERIOD block. The information specified in the PERIOD block applies only to that stress period.

\item \texttt{releasesetting}---specifies the steps at the start of which particles will be released.  The setting applies to all release points defined in PACKAGEDATA.

\begin{lstlisting}[style=blockdefinition]
ALL
FIRST
FREQUENCY <frequency>
STEPS <steps(<nstp)>
\end{lstlisting}

\item \texttt{ALL}---keyword to indicate release of particles at the start of all time steps in the period.

\item \texttt{FIRST}---keyword to indicate release of particles at the start of the first time step in the period. This keyword may be used in conjunction with other keywords to release particles at the start of multiple time steps.

\item \texttt{frequency}---release particles at the specified time step frequency. This keyword may be used in conjunction with other keywords to release particles at the start of multiple time steps.

\item \texttt{steps}---release particles at the start of each step specified in STEPS. This keyword may be used in conjunction with other keywords to release particles at the start of multiple time steps.

\end{description}

