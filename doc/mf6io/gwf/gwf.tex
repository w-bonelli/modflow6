This section describes the data files for a \mf Groundwater Flow (GWF) Model.  A GWF Model is added to the simulation by including a GWF entry in the MODELS block of the simulation name file.

There are three types of spatial discretization approaches that can be used with the GWF Model.  Input for a GWF Model may be entered in a structured form, like for previous MODFLOW versions, in that users specify cells using their layer, row, and column indices.  Users may also work with a layered grid in which cells are defined using vertices.  In this case, users specify cells using the layer number and the cell number.  Lastly, GWF Models may be entered as fully unstructured models, in which cells are specified using only their cell number.  Once a spatial discretization approach has been selected, then all input with cell indices must be entered accordingly.

The GWF Model is designed to permit input to be gathered, as it is needed, from many different files.  Likewise, results from the model calculations can be written to a number of output files. The GWF Model Listing File is a key file to which the GWF model output is written.  As \mf runs, information about the GWF Model is written to the GWF Model Listing File, including much of the input data (as a record of the simulation) and calculated results.  Details about the files used by each package are provided in this section on the GWF Model Instructions.

\mf is further designed to allow the user to control the amount, type, and frequency of information to be output. Much of the output will be written to the Simulation and GWF Model Listing Files, but some model output can be written to other files.  The Listing Files can become very large for common models.  Text editors are useful for examining the Listing File. The GWF Model Listing File includes a summary of the input data read for all packages.  In addition, the GWF Model Listing File optionally contains calculated head controlled by time step, and the overall volumetric budget controlled by time step. The Listing Files also contain information about solver convergence and error messages.  Output to other files can include head and cell-by-cell flow terms for use in calculations external to the model or in user-supplied applications such as plotting programs.

The GWF Model reads a file called the Name File, which specifies most of the files that will be used in a simulation. Several files are always required whereas other files are optional depending on the simulation. The Output Control Package receives instructions from the user to control the amount and frequency of output.  Details about the Name File and the Output Control Package are described in this section.

\subsection{Information for Existing MODFLOW Users}
\input{gwf/info_existing_users.tex}

\subsection{Array Input (READARRAY)}
Some GWF Model packages require arrays of information to be provided by the user.  This information is read using a generic READARRAY capability in \mf.  Within this user guide, variables that are read with READARRAY are marked accordingly, as shown in example input instructions for a DATA block.  

\begin{lstlisting}[style=blockdefinition]
BEGIN DATA
  ARRAY1
    <array1(nval)> -- READARRAY
END DATA
\end{lstlisting}

\noindent In this example, the uppercase ARRAY1 is a text string that is recognized by the program.  While reading through the DATA block, the program would recognize ARRAY1, and would then use READARRAY to fill \texttt{array1} with \texttt{nval} values.

\subsubsection{READARRAY Control Line}

READARRAY works similar to the array readers in previous MODFLOW versions.  It begins by reading a control line.  The control line has one of three forms shown below, and is limited to a length of 999 characters.

\begin{lstlisting}[style=blockdefinition]
1. CONSTANT <constant> 
\end{lstlisting}
With CONSTANT, all values in the array are set equal to \texttt{constant}. 

\begin{lstlisting}[style=blockdefinition]
2. INTERNAL [FACTOR <factor>] [IPRN <iprn>] 
\end{lstlisting}
With INTERNAL, the individual array elements will be read from the same file that contains the control line. 

\begin{lstlisting}[style=blockdefinition]
3. OPEN/CLOSE <fname> [FACTOR <factor>] [(BINARY)] [IPRN <iprn>]
\end{lstlisting}
With OPEN/CLOSE, the array will be read from the file whose name is specified by \texttt{fname}. This file will be opened just prior to reading the array and closed immediately after the array is read. A file that is read using this control line can contain only a single array. 

\subsubsection{READARRAY Variable Descriptions}

\begin{description}

\item \texttt{<constant>}---is a real number constant for real arrays and an integer constant for integer arrays. The \texttt{constant} value is assigned to the entire array. 

\item \texttt{FACTOR <factor>}---are a keyword and a real number factor for real arrays and an integer factor for integer arrays. The individual elements of the array are multiplied by \texttt{factor} after they are read. If \texttt{factor} is specified as 0, then it is changed to 1.

\item \texttt{(BINARY)}---is an option that indicates the OPEN/CLOSE file contains array data in binary (unformatted) form. A binary file that can be read by MODFLOW may be created in only two ways. The first way is to use MODFLOW to create the file by saving heads in a binary file. This is commonly done when the user desires to use computed heads from one simulation as initial heads for a subsequent simulation. The other way to create a binary file is to write a special program that generates a binary file.  ``(BINARY)'' can be specified only when the control line is OPEN/CLOSE.

\item \texttt{IPRN <iprn>}---are a keyword and a flag that indicates whether the array being read should be written to the Listing File after the array has been read and a code for indicating the format that should be used when the array is written. The format codes are the same as for MODFLOW-2005. IPRN is set to zero when the specified value exceeds those defined. If IPRN is less than zero or if the keyword and flag are omitted, the array will not be printed.  This IPRN capability is not functional for all data sets, and may be removed in future versions.

\end{description}

\begin{longtable}{p{2cm} p{2cm} p{2cm} p{2cm}}
\caption{IPRN Code and corresponding print formats for array readers.  These print codes determine how the user-provided array is written to the list file} 
\tabularnewline
\hline
\hline
\textbf{IPRN} & \textbf{Real} & \textbf{Integer} \\
\hline
\endhead
\hline
\endfoot
0 & 10G11.4 & 10I11 \\
1 & 11G10.3 & 60I1 \\
2 & 9G13.6 & 40I2 \\
3 & 15F7.1 & 30I3 \\
4 & 15F7.2 & 25I4 \\
5 & 15F7.3 & 20I5 \\
6 & 15F7.4 & 10I11 \\
7 & 20F5.0 & 25I2 \\
8 & 20F5.1 & 15I4 \\
9 & 20F5.2 & 10I6 \\
10 & 20F5.3 &  \\
11 & 20F5.4 &  \\
12 & 10G11.4 & \\
13 & 10F6.0 &  \\
14 & 10F6.1 &  \\
15 & 10F6.2 &  \\
16 & 10F6.3 &  \\
17 & 10F6.4 &  \\
18 & 10F6.5 &  \\
19 & 5G12.5 &  \\
20 & 6G11.4 &  \\
21 & 7G9.2 &  \\
%\label{table:ndim}
\end{longtable}


\subsubsection{READARRAY Examples}

The following examples use free-format control lines for reading an array. The example array is a real array consisting of 4 rows with 7 columns per row: 

\begin{lstlisting}[style=inputfile]
CONSTANT 5.7      This sets an entire array to the value "5.7". 
INTERNAL FACTOR 1.0 IPRN 3            This reads the array values from the 
 1.2 3.7 9.3 4.2 2.2 9.9 1.0      file that contains the control line. 
 3.3 4.9 7.3 7.5 8.2 8.7 6.6      Thus, the values immediately follow the 
 4.5 5.7 2.2 1.1 1.7 6.7 6.9      control line. 
 7.4 3.5 7.8 8.5 7.4 6.8 8.8 
OPEN/CLOSE inp.txt FACTOR 1.0 IPRN 3    Read array from formatted file "inp.dat". 
OPEN/CLOSE inp.bin FACTOR 1.0 (BINARY) IPRN 3     Read array from binary file "inp.bin". 
OPEN/CLOSE test.dat FACTOR 1.0 IPRN 3     Read array from file "test.dat". 
\end{lstlisting}


Some arrays define information that is required for the entire model grid, or part of a model grid.  This type of information is provided in a special type of data block called a ``GRIDDATA'' block.  For example, hydraulic conductivity is required for every cell in the model grid.  Hydraulic conductivity is read from a ``GRIDDATA'' block in the NPF Package input file.  For GRIDDATA arrays with one value for every cell in the model grid, the arrays can optionally be read in a LAYERED format, in which an array is provided for each layer of the grid.  Alternatively, the array can be read for the entire model grid.  As an example, consider the GRIDDATA block for the IC Package shown below:

\lstinputlisting[style=blockdefinition]{./mf6ivar/tex/gwf-ic-griddata.dat}

Here, the initial heads for the model are provided in the \texttt{strt} array.  If the optional LAYERED keyword is present, then a separate array is provided for each layer.  If the LAYERED keyword is not present, then the entire starting head array is read at once.  The LAYERED keyword may be useful to discretization packages of type DIS and DISV, which support the concept of layers.  Models defined with the DISU Package are not layered.

For a structured DIS model, the READARRAY utility is used to read arrays that are dimensioned to the full size of the grid (of size \texttt{nlay*nrow*ncol}). This utility first reads an array name, which associates the input to be read with the desired array.  For these arrays, an optional keyword ``LAYERED'' can be located next to the array name.  If ``LAYERED'' is detected, then a control line is provided for each layer and the array is filled with values for each model layer.  If the ``LAYERED'' keyword is absent, then a single control line is used and the entire array is filled at once.

For example, the following block shows one way the starting head array (STRT) could be specified for a model with 4 layers.  Following the array name and the ``LAYERED'' keyword are four control lines, one for each layer.

\begin{lstlisting}[style=inputfile]
  STRT LAYERED
     CONSTANT 10.0  #layer 1
     CONSTANT 10.0  #layer 2
     CONSTANT 10.0  #layer 3
     CONSTANT 10.0  #layer 4
\end{lstlisting}

In this next example, the ``LAYERED'' keyword is absent.  In this case, the control line applies to the entire \texttt{strt} array.  One control line is required, and a constant value of 10.0 will be assigned to STRT for all cells in the model grid.

\begin{lstlisting}[style=inputfile]
  STRT
     CONSTANT 10.0  #applies to all cells in the grid
\end{lstlisting}

\subsection{List Input}
Some items consist of several variables, such as layer, row, column, stage, and conductance, for example.  List input refers to a block of data with a separate item on each line.  For some common list types, the first set of variables is a cell identifier (denoted as \texttt{cellid} in this guide), such as layer, row, and column. With lists, the input data for each item must start on a new line. All variables for an item are assumed to be contained in a single line.  Each input variable has a data type, which can be Double Precision, Integer, or Character. Integers are whole numbers and must not include a decimal point or exponent. Double Precision numbers can include a decimal point and an exponent. If no decimal point is included in the entered value, then the decimal point is assumed to be at the right side of the value. Any printable character is allowed for character variables. 

Variables starting with the letters I-N are most commonly integers; however, in some instances, a character string may start with the letters I-N. Variables starting with the letters A-H and O-Z are primarily double precision numbers; however, these variable names may also be used for character data.  In \mf all variables are explicitly declared within the source code, as opposed to the implicit type declaration in previous MODFLOW versions.  This explicit declaration means that the variable type can be easily determined from the source code.

Free formatting is used throughout the input instructions.  With free format, values are not required to occupy a fixed number of columns in a line. Each value can occupy one or more columns as required to represent the value; however, the values must still be included in the prescribed order. One or more spaces, or a single comma optionally combined with spaces, must separate adjacent values. Also, a numeric value of zero must be explicitly represented with 0 and not by one or more spaces when free format is used, because detecting the difference between a space that represents 0 and a space that represents a value separator is not possible. Free format is similar to Fortran's list directed input.

Two capabilities included in Fortran's list-directed input are not included in the free-format input implemented in \mf. Null values in which input values are left unchanged from their previous values are not allowed. In general, MODFLOW's input values are not defined prior to their input.  A ``/'' cannot be used to terminate an input line without including values for all the variables; data values for all required input variables must be explicitly specified on an input line.  For character data, MODFLOW's free format implementation is less stringent than the list-directed input of Fortran. Fortran requires character data to be delineated by apostrophes. MODFLOW does not require apostrophes unless a blank or a comma is part of a character variable.

As an example of a list, consider the PERIOD block for the GHB Package.  The input format is  shown below:

\lstinputlisting[style=blockdefinition]{./mf6ivar/tex/gwf-ghb-period.dat}

Each line represents a separate item, which consists of variables.  In this case, the first variable of the item, \texttt{cellid} is an array of size \texttt{ncelldim}.  The next two variables of the item are \texttt{bhead} and \texttt{cond}.  Lastly, the item has two optional variables, \texttt{aux} and \texttt{boundname}.  Three of the variables shown in the list are colored in blue.  Variables that are colored in blue mean that they can be represented with a time series.  The time series capability is described in the section on Time-Variable Input in this document.  

The following is simple example of a PERIOD block for the GHB Package, which shows how a list is entered by the user.

\begin{lstlisting}[style=inputfile]
BEGIN PERIOD 1
#      lay       row       col     stage      cond
         1        13         1     988.0     0.038
         1        14         9    1045.0     0.038
END PERIOD
\end{lstlisting}

As described earlier in the section on ``Block and Keyword Input,'' block information can be read from a separate text file.  To activate reading a list from separate text file, the first and only entry in the block must be a control line of the following form:  

\begin{lstlisting}[style=blockdefinition]
  OPEN/CLOSE <fname>
\end{lstlisting}

\noindent where \texttt{fname} is the name of the file containing the list.  Lists for the stress packages (CHD, WEL, DRN, RIV, GHB, RCH, and EVT) have an additional BINARY option.  The BINARY  option is not supported for the advanced stress packages (LAK, MAW, SFR, UZF).  The BINARY options is specified as follows:

\begin{lstlisting}[style=blockdefinition]
  OPEN/CLOSE <fname> [(BINARY)]
\end{lstlisting}

If the (BINARY) keyword is found on the control line, then the file is opened as an unformatted file on unit 99, and the list is read.  There are a number of requirements for using the (BINARY) option for lists.  All stress package lists begin with integer values for the \texttt{cellid} (layer, row, and column, for example).  These values must be represented as integer numbers in the unformatted file.  Also, all auxiliary data must be included in the binary file; auxiliary data must be represented as double precision numbers.  Lastly, the (BINARY) option does not support entry of \texttt{boundname}, and so the BOUNDNAMES option should not be activated in the OPTIONS block for the package.  

\subsection{Units of Length and Time}
The GWF Model formulates the groundwater flow equation without using prescribed length and time units. Any consistent units of length and time can be used when specifying the input data for a simulation. This capability gives a certain amount of freedom to the user, but care must be exercised to avoid mixing units.  The program cannot detect the use of inconsistent units.  For example, if hydraulic conductivity is entered in units of feet per day and pumpage as cubic meters per second, the program will run, but the results will be meaningless. Other processes generally are expected to work with consistent length and time units; however, other processes could conceivably place restrictions on which units are supported.

The user can set flags that specify the length and time units (see the input instructions for the Timing Module and Spatial Discretization Files), which may be useful in various parts of MODFLOW.  For example, the program will label the table of simulation time with time units if the time units are specified by the optional TIME\_UNITS label, which can be set in the TDIS Package.  If the time units are not specified, the program still runs, but the table of simulation time does not indicate the time units. An optional LENGTH\_UNITS label can be set in the Discretization Package. Situations in other processes may require that the length or time units be specified.  In such situations, the input instructions will state the requirements. Remember that specifying the unit flags does not enforce consistent use of units.  The user must insure that consistent units are used in all input data.

\subsection{Steady-State Simulations}
A steady-state simulation is represented by a single stress period having a single time step with the storage term set to zero. Setting the number and length of stress periods and time steps is the responsibility of the Timing Module of the \mf framework. The length of the stress period and time step will not affect the head solution because the time derivative is not calculated in a steady-state problem. Setting the storage term to zero is the responsibility of the Storage Package. Most other packages need not "know" that a simulation is steady state.

A GWF Model also can be mixed transient and steady state because each stress period can be designated transient or steady state.  Thus, a GWF Model can start with a steady-state stress period and continue with one or more transient stress periods.  The settings for controlling steady-state and transient options are in the Storage Package.  If the Storage Package is not specified for a GWF Model, then the storage terms are zero and the GWF Model will be steady state.

\subsection{Volumetric Budget}
A summary of all inflows (sources) and outflows (sinks) of water is called a water budget.  The water budget for the GWF Model is termed a volumetric budget because volumes of water and volumetric flow rates are involved; thus strictly speaking, a volumetric budget is not a mass balance, although this term has been used in other model reports.  \mf calculates a water budget for the overall model as a check on the acceptability of the solution, and to provide a summary of the sources and sinks of water to the flow system.  The water budget is printed to the GWF Model Listing File for selected time steps.

Numerical solution techniques for simultaneous equations do not always result in a correct answer; in particular, iterative solvers may stop iterating before a sufficiently close approximation to the solution is attained.  A water budget provides an indication of the overall acceptability of the solution.  The system of equations solved by the model actually consists of a flow continuity statement for each model cell.  Continuity should also exist for the total flows into and out of the model---that is, the difference between total inflow and total outflow should equal the total change in storage.  In the model program, the water budget is calculated independently of the equation solution process, and in this sense may provide independent evidence of a valid solution.

The total budget as printed in the output does not include internal flows between model cells---only flows into or out of the model as a whole. For example, flow to or from rivers, flow to or from constant-head cells, and flow to or from wells are all included in the overall budget terms.  Flow into and out of storage is also considered part of the overall budget inasmuch as accumulation in storage effectively removes water from the flow system and storage release effectively adds water to the flow---even though neither process, in itself, involves the transfer of water into or out of the ground-water regime.  Each hydrologic package calculates its own contribution to the budget.

For every time step, the budget subroutine of each hydrologic package calculates the rate of flow into and out of the system due to the process simulated by the package.  The inflows and outflows for each component of flow are stored separately.  Most packages deal with only one such component of flow.  In addition to flow, the volumes of water entering and leaving the model during the time step are calculated as the product of flow rate and time-step length.  Cumulative volumes, from the beginning of the simulation, are then calculated and stored.

The GWF Model uses the inflows, outflows, and cumulative volumes to write the budget to the Listing File at the times requested by the model user.  When a budget is written, the flow rates for the last time step and cumulative volumes from the beginning of simulation are written for each component of flow.  Inflows are written separately from outflows.  Following the convention indicated above, water entering storage is treated as an outflow (that is, as a loss of water from the flow system) while water released from storage is treated as an inflow (that is, a source of water to the flow system).  In addition, total inflow and total outflow are written, as well as the difference between total inflow and outflow.  The difference is then written as a percentage error, calculated using the formula:

\begin{equation}
D = \frac{100 (IN-OUT)}{(IN + OUT) / 2}
\end{equation}

\noindent where $D$ is the percentage error term, $IN$ is the total inflow to the system, and $OUT$ is the total outflow.

If the model equations are solved correctly, the percentage error should be small.  In general, flow rates may be taken as an indication of solution validity for the time step to which they apply, while cumulative volumes are an indication of validity for the entire simulation up to the time of the output.  The budget is written to the GWF Model Listing File at the end of each stress period whether requested or not.

\subsection{Cell-By-Cell Flows}
In some situations, calculating flow terms for various subregions of the model is useful.  To facilitate such calculations, provision has been made to save flow terms for individual cells in a separate binary file so they can be used in computations external to the model itself.  These individual cell flows are referred to here as ``cell-by-cell'' flow terms and are of four general types: (1) cell-by-cell stress flows, or flows into or from an individual cell caused by one of the external stresses represented in the model, such as evapotranspiration or recharge; (2) cell-by-cell storage terms, which give the rate of accumulation or depletion of storage in an individual cell; and (3) internal cell-by-cell flows, which are actually the flows across individual cell faces---that is, between adjacent model cells.  These four kinds of cell-by-cell flow terms are discussed further in subsequent paragraphs.  To save any of these cell-by-cell terms, two flags in the model input must be set.  The input to the Output Control file indicates the time steps for which cell-by-cell terms are to be saved. In addition, each hydrologic package includes an option called SAVE\_FLOWS that must be set if the cell-by-cell terms computed by that package are to be saved.  Thus, if the appropriate option in the Evapotranspiration Package input is set, cell-by-cell evapotranspiration terms will be saved for each time step for which the saving of cell-by-cell flow is requested through the Output Control Option.  Only flow values are saved in the cell-by-cell files; neither water volumes nor cumulative water volumes are included.  The flow dimensions are volume per unit time, where volume and time are in the same units used for all model input data.  The cell-by-cell flow values are stored in unformatted form to make the most efficient use of disk space; see the Budget File section toward the end of this user guide for information on how the data are written to a file.

The cell-by-cell storage term gives the net flow to or from storage in a variable-head cell.  The net storage for each cell in the grid is saved in transient simulations if the appropriate flags are set.  Withdrawal from storage in the cell is considered positive, whereas accumulation in storage is considered negative.

The cell-by-cell constant-head flow term gives the flow into or out of an individual constant-head cell (specified with the CHD Package).  This term is always associated with the constant-head cell itself, rather than with the surrounding cells that contribute or receive the flow.  A constant-head cell may be surrounded by as many as six adjacent variable-head cells for a regular grid or any number of cells for the other grid types.  The cell-by-cell calculation provides a single flow value for each constant-head cell, representing the algebraic sum of the flows between that cell and all of the adjacent variable-head cells.  A positive value indicates that the net flow is away from the constant-head cell (into the variable-head part of the grid); a negative value indicates that the net flow is into the constant-head cell.

The internal cell-by-cell flow values represent flows across the individual faces of a model cell.  Flows between cells are written in the compressed row storage format, whereby the flow between cell $n$ and each one of its connecting $m$ neighbor cells are contained in a single one-dimensional array.  Flows are positive for the cell in question.  Thus the flow reported for cell $n$ and its connection with cell $m$ is opposite in sign to the flow reported for cell $m$ and its connection with cell $n$.  These internal cell-by-cell flow values are useful in calculations of the groundwater flow into various subregions of the model, or in constructing flow vectors.

Cell-by-cell stress flows are flow rates into or out of the model, at a particular cell, owing to one particular external stress.  For example, the cell-by-cell evapotranspiration term for cell $n$ would give the flow out of the model by evapotranspiration from cell $n$.  Cell-by-cell stress flows are considered positive if flow is into the cell, and negative if out of the cell.

\newpage
\subsection{GWF Model Name File}
The PRT Model Name File specifies the options and packages that are active for a PRT model.  The Name File contains two blocks: OPTIONS  and PACKAGES. The length of each line must be 299 characters or less. The lines in each block can be in any order.  Files listed in the PACKAGES block must exist when the program starts. 

Comment lines are indicated when the first character in a line is one of the valid comment characters.  Commented lines can be located anywhere in the file. Any text characters can follow the comment character. Comment lines have no effect on the simulation; their purpose is to allow users to provide documentation about a particular simulation. 

\vspace{5mm}
\subsubsection{Structure of Blocks}
\lstinputlisting[style=blockdefinition]{./mf6ivar/tex/prt-nam-options.dat}
\lstinputlisting[style=blockdefinition]{./mf6ivar/tex/prt-nam-packages.dat}

\vspace{5mm}
\subsubsection{Explanation of Variables}
\begin{description}
% DO NOT MODIFY THIS FILE DIRECTLY.  IT IS CREATED BY mf6ivar.py 

\item \textbf{Block: OPTIONS}

\begin{description}
\item \texttt{list}---is name of the listing file to create for this PRT model.  If not specified, then the name of the list file will be the basename of the PRT model name file and the '.lst' extension.  For example, if the PRT name file is called ``my.model.nam'' then the list file will be called ``my.model.lst''.

\item \texttt{PRINT\_INPUT}---keyword to indicate that the list of all model stress package information will be written to the listing file immediately after it is read.

\item \texttt{PRINT\_FLOWS}---keyword to indicate that the list of all model package flow rates will be printed to the listing file for every stress period time step in which ``BUDGET PRINT'' is specified in Output Control.  If there is no Output Control option and ``PRINT\_FLOWS'' is specified, then flow rates are printed for the last time step of each stress period.

\item \texttt{SAVE\_FLOWS}---keyword to indicate that all model package flow terms will be written to the file specified with ``BUDGET FILEOUT'' in Output Control.

\end{description}
\item \textbf{Block: PACKAGES}

\begin{description}
\item \texttt{ftype}---is the file type, which must be one of the following character values shown in table~\ref{table:ftype}. Ftype may be entered in any combination of uppercase and lowercase.

\item \texttt{fname}---is the name of the file containing the package input.  The path to the file should be included if the file is not located in the folder where the program was run.

\item \texttt{pname}---is the user-defined name for the package. PNAME is restricted to 16 characters.  No spaces are allowed in PNAME.  PNAME character values are read and stored by the program for stress packages only.  These names may be useful for labeling purposes when multiple stress packages of the same type are located within a single PRT Model.  If PNAME is specified for a stress package, then PNAME will be used in the flow budget table in the listing file; it will also be used for the text entry in the cell-by-cell budget file.  PNAME is case insensitive and is stored in all upper case letters.

\end{description}


\end{description}

\begin{table}[H]
\caption{Ftype values described in this report.  The \texttt{Pname} column indicates whether or not a package name can be provided in the name file.  The capability to provide a package name also indicates that the PRT Model can have more than one package of that Ftype}
\small
\begin{center}
\begin{tabular*}{\columnwidth}{l l l}
\hline
\hline
Ftype & Input File Description & \texttt{Pname}\\
\hline
DIS6 & Rectilinear Discretization Input File \\
DISV6 & Discretization by Vertices Input File \\
MIP6 & Model Input File \\
FMI6 & Flow Model Interface Package &  \\ 
PRP6 & Particle Release Point Package \\
OC6 & Output Control Option \\
OBS6 & Observations Option \\
\hline 
\end{tabular*}
\label{table:ftypeprt}
\end{center}
\normalsize
\end{table}

\vspace{5mm}
\subsubsection{Example Input File}
\lstinputlisting[style=inputfile]{./mf6ivar/examples/prt-nam-example.dat}



\newpage
\subsection{Structured Discretization (DIS) Input File}
\input{gwf/dis}

\newpage
\subsection{Discretization by Vertices (DISV) Input File}
\input{gwf/disv}

\newpage
\subsection{Unstructured Discretization (DISU) Input File}
\input{gwf/disu}

\newpage
\subsection{Initial Conditions (IC) Package}
\input{gwf/ic}

\newpage
\subsection{Output Control (OC) Option}
Input to the Output Control Option of the Particle Tracking Model is read from the file that is specified as type ``OC6'' in the Name File. If no ``OC6'' file is specified, default output control is used. The Output Control Option determines how and when concentrations are printed to the listing file and/or written to a separate binary output file.  Under the default, concentration and overall transport budget are written to the Listing File at the end of every stress period. The default printout format for concentrations is 10G11.4.  The concentrations and overall transport budget are also written to the list file if the simulation terminates prematurely due to failed convergence.

Output Control data must be specified using words.  The numeric codes supported in earlier MODFLOW versions can no longer be used.

For the PRINT and SAVE options of concentration, there is no option to specify individual layers.  Whenever the concentration array is printed or saved, all layers are printed or saved.

\vspace{5mm}
\subsubsection{Structure of Blocks}
\vspace{5mm}

\noindent \textit{FOR EACH SIMULATION}
\lstinputlisting[style=blockdefinition]{./mf6ivar/tex/prt-oc-options.dat}
\vspace{5mm}
\noindent \textit{FOR ANY STRESS PERIOD}
\lstinputlisting[style=blockdefinition]{./mf6ivar/tex/prt-oc-period.dat}

\vspace{5mm}
\subsubsection{Explanation of Variables}
\begin{description}
% DO NOT MODIFY THIS FILE DIRECTLY.  IT IS CREATED BY mf6ivar.py 

\item \textbf{Block: OPTIONS}

\begin{description}
\item \texttt{BUDGET}---keyword to specify that record corresponds to the budget.

\item \texttt{FILEOUT}---keyword to specify that an output filename is expected next.

\item \texttt{budgetfile}---name of the output file to write budget information.

\item \texttt{BUDGETCSV}---keyword to specify that record corresponds to the budget CSV.

\item \texttt{budgetcsvfile}---name of the comma-separated value (CSV) output file to write budget summary information.  A budget summary record will be written to this file for each time step of the simulation.

\item \texttt{TRACK}---keyword to specify that record corresponds to a binary track file.

\item \texttt{trackfile}---name of the output file to write tracking information.

\item \texttt{TRACKCSV}---keyword to specify that record corresponds to a CSV track file.

\item \texttt{trackcsvfile}---name of the comma-separated value (CSV) file to write tracking information.

\item \texttt{TRACK\_ALL}---keyword to indicate that ...

\item \texttt{TRACK\_RELEASE}---keyword to indicate that particle tracking output is to be written when a particle is released

\item \texttt{TRACK\_TRANSIT}---keyword to indicate that particle tracking output is to be written when a particle exits a cell

\item \texttt{TRACK\_TIMESTEP}---keyword to indicate that particle tracking output is to be written at the end of each time step

\item \texttt{TRACK\_TERMINATE}---keyword to indicate that particle tracking output is to be written when a particle terminates for any reason

\item \texttt{TRACK\_WEAKSINK}---keyword to indicate that particle tracking output is to be written when a particle exits a weak sink (a cell which removes some but not all inflow from adjacent cells)

\item \texttt{TRACK\_USERTIME}---keyword to indicate that particle tracking output is to be written at user-specified times, provided as double precision values to the TRACK\_TIMES or TRACK\_TIMESFILE options

\item \texttt{TRACK\_TIMES}---keyword indicating tracking times will follow

\item \texttt{times}---times to track, relative to the beginning of the simulation.

\item \texttt{TRACK\_TIMESFILE}---keyword indicating tracking times file name will follow

\item \texttt{timesfile}---name of the tracking times file

\end{description}
\item \textbf{Block: PERIOD}

\begin{description}
\item \texttt{iper}---integer value specifying the starting stress period number for which the data specified in the PERIOD block apply.  IPER must be less than or equal to NPER in the TDIS Package and greater than zero.  The IPER value assigned to a stress period block must be greater than the IPER value assigned for the previous PERIOD block.  The information specified in the PERIOD block will continue to apply for all subsequent stress periods, unless the program encounters another PERIOD block.

\item \texttt{SAVE}---keyword to indicate that information will be saved this stress period.

\item \texttt{PRINT}---keyword to indicate that information will be printed this stress period.

\item \texttt{rtype}---type of information to save or print.  Can only be BUDGET.

\item \texttt{ocsetting}---specifies the steps for which the data will be saved.

\begin{lstlisting}[style=blockdefinition]
ALL
FIRST
LAST
FREQUENCY <frequency>
STEPS <steps(<nstp)>
\end{lstlisting}

\item \texttt{ALL}---keyword to indicate save for all time steps in period.

\item \texttt{FIRST}---keyword to indicate save for first step in period. This keyword may be used in conjunction with other keywords to print or save results for multiple time steps.

\item \texttt{LAST}---keyword to indicate save for last step in period. This keyword may be used in conjunction with other keywords to print or save results for multiple time steps.

\item \texttt{frequency}---save at the specified time step frequency. This keyword may be used in conjunction with other keywords to print or save results for multiple time steps.

\item \texttt{steps}---save for each step specified in STEPS. This keyword may be used in conjunction with other keywords to print or save results for multiple time steps.

\end{description}


\end{description}

\vspace{5mm}
\subsubsection{Example Input File}
\lstinputlisting[style=inputfile]{./mf6ivar/examples/prt-oc-example.dat}


\newpage
\subsection{Observation (OBS) Utility for a GWF Model}
\input{gwf/gwf-obs}

\newpage
\subsection{Node Property Flow (NPF) Package}
\input{gwf/npf}

\newpage
\subsection{Time-Varying Hydraulic Conductivity (TVK) Package}
\input{gwf/tvk}

\newpage
\subsection{Horizontal Flow Barrier (HFB) Package}
\input{gwf/hfb}

\newpage
\subsection{Storage (STO) Package}
\input{gwf/sto}

\newpage
\subsection{Time-Varying Storage (TVS) Package}
\input{gwf/tvs}

\newpage
\subsection{Skeletal Storage, Compaction, and Subsidence (CSUB) Package}
\input{gwf/csub}

\newpage
\subsection{Buoyancy (BUY) Package}
\input{gwf/buy}

\newpage
\subsection{Viscosity (VSC) Package}
Input to the Viscosity (VSC) Package is read from the file that has type ``VSC6'' in the Name File.  If the VSC Package is active within a groundwater flow model, then the model will account for the dependence of fluid viscosity on solute concentration and the resulting changes in hydraulic conductivity and stress-package conductances, which vary inversely with viscosity.  Viscosity can be calculated as a function of one or more groundwater solute transport (GWT) species using an approach described in the Supplemental Technical Information document distributed with MODFLOW 6 (Chapter 8).  Only one VSC Package can be specified for a GWF model. The VSC Package can be coupled with one or more GWT Models so that the fluid viscosity is updated dynamically with one or more simulated concentration fields.

The VSC Package calculates fluid viscosity using the following equation from \cite{langevin2008seawat}:

\begin{equation}
\label{eqn:visclinear}
\mu = VISCREF + \sum_{i=1}^{NVISCSPECIES} DVISCDC_i \left ( CONCENTRATION_i - CVISCREF_i \right )
\end{equation}

\noindent where $\mu$ is the calculated viscosity, $VISCREF$ is the viscosity of a reference fluid, typically taken to be freshwater at a temperature of 20 degrees Celsius, $NVISCSPECIES$ is the number of chemical species that contribute to the viscosity calculation, $DVISCDC_i$ is the parameter that describes how viscosity changes linearly as a function of concentration for chemical species $i$ (i.e. the slope of a line that relates viscosity to concentration), $CONCENTRATION_i$ is the concentration of species $i$, and $CVISCREF_i$ is the reference concentration for species $i$ corresponding to when the viscosity of the reference fluid is equal to $VISCREF$, which is normally set to a concentration of zero.

In many applications, variations in temperature have a greater effect on fluid viscosity than variations in solute concentration. When a GWT model is formulated such that one of the transported ``species'' is heat (thermal energy), with ``concentration'' used to represent temperature \citep{zheng2010supplemental}, the viscosity can vary linearly with temperature, as it can with any other ``concentration.''  In that case, $CONCENTRATION_i$ and $CVISCREF_i$ represent the simulated and reference temperatures, respectively, and $DVISCDC_i$ represents the rate at which viscosity changes with temperature. In addition, the viscosity formula can optionally include a nonlinear dependence on temperature. In that case, equation 3 becomes

\begin{equation}
\label{eqn:viscnonlinear}
\mu = \mu_T(T) + \sum_{i=1}^{NVISCSPECIES} DVISCDC_i \left ( CONCENTRATION_i - CVISCREF_i \right )
\end{equation}

\noindent where the first term on the right-hand side, $\mu_T(T)$, is a nonlinear function of temperature, and the summation corresponds to the summation in equation \ref{eqn:visclinear}, in which one of the ``species'' is heat. The nonlinear term in equation \ref{eqn:viscnonlinear} is of the form

\begin{equation}
\label{eqn:munonlinear}
\mu_T(T) = CVISCREF_i \cdot A_2^{\left [ \frac {-A_3 \left ( CONCENTRATION_i - CVISCREF_i \right ) } {\left ( CONCENTRATION_i + A_4 \right ) \left ( CVISCREF_i + A_4 \right )} \right ]}
\end{equation}

\noindent where the coefficients $A_2$, $A_3$, and $A_4$ are specified by the user.  Values for $A_2$, $A_3$, and $A_4$ are commonly 10, 248.7, and 133.15, respectively  \citep{langevin2008seawat, Voss1984sutra}.
 
\subsubsection{Stress Packages}

For head-dependent stress packages, the VSC Package can adjust the conductance used to calculate flow between the boundary and the aquifer to account for variations in viscosity. Conductance is assumed to vary inversely with viscosity.

By default, the boundary viscosity is set equal to VISCREF, which, for freshwater, is typically set equal to 1.0. However, there are two additional options for setting the viscosity of a boundary package.  The first is to assign an auxiliary variable with the name ``VISCOSITY''.  If an auxiliary variable named ``VISCOSITY'' is detected, then it will be assigned as the viscosity of the fluid entering from the boundary.  Alternatively, a viscosity value can be calculated for each boundary using the viscosity equation described above and one or more concentrations provided as auxiliary variables.  In this case, the user must assign one auxiliary variable for each AUXSPECIESNAME listed in the PACKAGEDATA block below.  Thus, there must be NVISCSPECIES auxiliary variables, each with the identical name as those specified in PACKAGEDATA.  The VSC Package will calculate the viscosity for each boundary using these concentrations and the values specified for VISCREF, DVISCDC, and CVISCREF.  If the boundary package contains an auxiliary variable named VISCOSITY and also contains AUXSPECIESNAME auxiliary variables, then the boundary viscosity value will be assigned to the one in the VISCOSITY auxiliary variable.

A GWT Model can be used to calculate concentrations for the advanced stress packages (LAK, SFR, MAW, and UZF) if corresponding advanced transport packages are specified (LKT, SFT, MWT, and UZT).  The advanced stress packages have an input option called FLOW\_PACKAGE\_AUXILIARY\_NAME.  When activated, this option will result in the simulated concentration for a lake or other feature being copied from the advanced transport package into the auxiliary variable for the corresponding GWF stress package.  This means that the viscosity for a lake or stream, for example, can be dynamically updated during the simulation using concentrations from advanced transport packages that are fed into auxiliary variables in the advanced stress packages, and ultimately used by the VSC Package to calculate a fluid viscosity.  This concept also applies when multiple GWT Models are used simultaneously to simulate multiple species.  In this case, multiple auxiliary variables are required for an advanced stress package, with each one representing a concentration from a different GWT Model.  


\begin{longtable}{p{3cm} p{12cm}}
\caption{Description of viscosity terms for stress packages}
\tabularnewline
\hline
\hline
\textbf{Stress Package} & \textbf{Note} \\
\hline
\endhead
\hline
\endfoot
GHB & A VISCOSITY auxiliary variable or one or more auxiliary variables for calculating viscosity in the equation of state can be specified \\
RIV & A VISCOSITY auxiliary variable or one or more auxiliary variables for calculating viscosity in the equation of state can be specified \\
DRN & The drain formulation assumes that the drain boundary contains water of the same viscosity as the discharging water; auxiliary variables have no effect on the drain calculation  \\
LAK & A VISCOSITY auxiliary variable or one or more auxiliary variables for calculating viscosity in the equation of state can be specified \\
SFR & A VISCOSITY auxiliary variable or one or more auxiliary variables for calculating viscosity in the equation of state can be specified \\
MAW & A VISCOSITY auxiliary variable or one or more auxiliary variables for calculating viscosity in the equation of state can be specified \\
UZF & Viscosity variations not implemented \\
\end{longtable}

\vspace{5mm}
\subsubsection{Structure of Blocks}

\vspace{5mm}
\noindent \textit{FOR EACH SIMULATION}
\lstinputlisting[style=blockdefinition]{./mf6ivar/tex/gwf-vsc-options.dat}
\lstinputlisting[style=blockdefinition]{./mf6ivar/tex/gwf-vsc-dimensions.dat}
\lstinputlisting[style=blockdefinition]{./mf6ivar/tex/gwf-vsc-packagedata.dat}

\vspace{5mm}
\subsubsection{Explanation of Variables}
\begin{description}
% DO NOT MODIFY THIS FILE DIRECTLY.  IT IS CREATED BY mf6ivar.py 

\item \textbf{Block: OPTIONS}

\begin{description}
\item \texttt{viscref}---fluid reference viscosity used in the equation of state.  This value is set to 1.0 if not specified as an option.

\item \texttt{temperature\_species\_name}---string used to identify the auxspeciesname in PACKAGEDATA that corresponds to the temperature species.  There can be only one occurrence of this temperature species name in the PACKAGEDATA block or the program will terminate with an error.  This value has no effect if viscosity does not depend on temperature.

\item \texttt{thermal\_formulation}---may be used for specifying which viscosity formulation to use for the temperature species. Can be either LINEAR or NONLINEAR. The LINEAR viscosity formulation is the default.

\item \texttt{thermal\_a2}---is an empirical parameter specified by the user for calculating viscosity using a nonlinear formulation.  If A2 is not specified, a default value of 10.0 is assigned (Voss, 1984).

\item \texttt{thermal\_a3}---is an empirical parameter specified by the user for calculating viscosity using a nonlinear formulation.  If A3 is not specified, a default value of 248.37 is assigned (Voss, 1984).

\item \texttt{thermal\_a4}---is an empirical parameter specified by the user for calculating viscosity using a nonlinear formulation.  If A4 is not specified, a default value of 133.15 is assigned (Voss, 1984).

\item \texttt{VISCOSITY}---keyword to specify that record corresponds to viscosity.

\item \texttt{FILEOUT}---keyword to specify that an output filename is expected next.

\item \texttt{viscosityfile}---name of the binary output file to write viscosity information.  The viscosity file has the same format as the head file.  Viscosity values will be written to the viscosity file whenever heads are written to the binary head file.  The settings for controlling head output are contained in the Output Control option.

\end{description}
\item \textbf{Block: DIMENSIONS}

\begin{description}
\item \texttt{nviscspecies}---number of species used in the viscosity equation of state.  If either concentrations or temperature (or both) are used to update viscosity then then nrhospecies needs to be at least one.

\end{description}
\item \textbf{Block: PACKAGEDATA}

\begin{description}
\item \texttt{iviscspec}---integer value that defines the species number associated with the specified PACKAGEDATA data entered on each line. IVISCSPECIES must be greater than zero and less than or equal to NVISCSPECIES. Information must be specified for each of the NVISCSPECIES species or the program will terminate with an error.  The program will also terminate with an error if information for a species is specified more than once.

\item \texttt{dviscdc}---real value that defines the slope of the line defining the linear relationship between viscosity and temperature or between viscosity and concentration, depending on the type of species entered on each line.  If the value of AUXSPECIESNAME entered on a line corresponds to TEMPERATURE\_SPECIES\_NAME (in the OPTIONS block), this value will be used when VISCOSITY\_FUNC is equal to LINEAR (the default) in the OPTIONS block.  When VISCOSITY\_FUNC is set to NONLINEAR, a value for DVISCDC must be specified though it is not used.

\item \texttt{cviscref}---real value that defines the reference temperature or reference concentration value used for this species in the viscosity equation of state.  If AUXSPECIESNAME entered on a line corresponds to TEMPERATURE\_SPECIES\_NAME (in the OPTIONS block), then CVISCREF refers to a reference temperature, otherwise it refers to a reference concentration.

\item \texttt{modelname}---name of a GWT model used to simulate a species that will be used in the viscosity equation of state.  This name will have no effect if the simulation does not include a GWT model that corresponds to this GWF model.

\item \texttt{auxspeciesname}---name of an auxiliary variable in a GWF stress package that will be used for this species to calculate the viscosity values.  If a viscosity value is needed by the Viscosity Package then it will use the temperature or concentration values associated with this AUXSPECIESNAME in the viscosity equation of state.  For advanced stress packages (LAK, SFR, MAW, and UZF) that have an associated advanced transport package (LKT, SFT, MWT, and UZT), the FLOW\_PACKAGE\_AUXILIARY\_NAME option in the advanced transport package can be used to transfer simulated temperature or concentration(s) into the flow package auxiliary variable.  In this manner, the Viscosity Package can calculate viscosity values for lakes, streams, multi-aquifer wells, and unsaturated zone flow cells using simulated concentrations.

\end{description}


\end{description}

\vspace{5mm}
\subsubsection{Example Input File}
\lstinputlisting[style=inputfile]{./mf6ivar/examples/gwf-vsc-example.dat}


\newpage
\subsection{Constant-Head (CHD) Package}
\input{gwf/chd}

\newpage
\subsection{Well (WEL) Package}
\input{gwf/wel}

\newpage
\subsection{Drain (DRN) Package}
\input{gwf/drn}

\newpage
\subsection{River (RIV) Package}
\input{gwf/riv}

\newpage
\subsection{General-Head Boundary (GHB) Package}
\input{gwf/ghb}

\newpage
\subsection{Recharge (RCH) Package -- List-Based Input}
\input{gwf/rch}

\newpage
\subsection{Recharge (RCH) Package -- Array-Based Input}
\input{gwf/rcha}

\newpage
\subsection{Evapotranspiration (EVT) Package -- List-Based Input}
\input{gwf/evt}

\newpage
\subsection{Evapotranspiration (EVT) Package -- Array-Based Input}
\input{gwf/evta}

\newpage
\subsection{Multi-Aquifer Well (MAW) Package}
\input{gwf/maw}

\newpage
\subsection{Streamflow Routing (SFR) Package}
\input{gwf/sfr}

\newpage
\subsection{Lake (LAK) Package}
\input{gwf/lak}

\newpage
\subsection{Unsaturated Zone Flow (UZF) Package}
\input{gwf/uzf}

\newpage
\subsection{Water Mover (MVR) Package}
\input{gwf/mvr}

\newpage
\subsection{Ghost-Node Correction (GNC) Package}
\input{gwf/gnc}

\newpage
\subsection{Groundwater Flow (GWF) Exchange}
\input{gwf/gwf-gwf}

