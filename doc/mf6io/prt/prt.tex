The PRT Model performs three-dimensional particle tracking in flowing groundwater. ...  

This section describes the data files for a \mf Particle Tracking (PRT) Model.  A PRT Model is added to the simulation by including a PRT entry in the MODELS block of the simulation name file.  There are currently two types of spatial discretization approaches that can be used with the PRT Model: DIS and DISV.  The input instructions for these three packages are not described here in this section on PRT Model input; input instructions for these three packages are described in the section on GWF Model input.

The PRT Model is designed to permit input to be gathered, as it is needed, from many different files.  Likewise, results from the model calculations can be written to a number of output files. ...  Details about the files used by each package are provided in this section on the PRT Model Instructions.

The PRT Model reads a file called the Name File, which specifies most of the files that will be used in a simulation. Several files are always required whereas other files are optional depending on the simulation. The Output Control Package receives instructions from the user to control the amount and frequency of output.  Details about the Name File and the Output Control Package are described in this section.

For the PRT Model, ``flows'' (unless stated otherwise) represent particle mass ``flow'' in mass per time, rather than groundwater flow.  In this implementation, each particle is assigned unit mass, and the numerical value of the flow can be interpreted as particles per time.

\begin{enumerate}

\item The PRT Model simulates transport of ...; however, because \mf allows for multiple models of the same type to be included in a single simulation, ... can be represented by using multiple PRT Models.

\item The PRT Model requires simulated groundwater flows as input. Simulated flows from the GWF Model can be passed in memory to the PRT Model in the same simulation via a GWF-PRT Exchange.  Alternatively, the PRT Model can read binary flow and head files saved by a previously run GWF Model.  The current implemention of the PRT Model does not support particle tracking through the advanced stress packages or the Water Mover Package.

\item Although there is GWF-GWF Exchange, a PRT-PRT Exchange has not yet been developed to connect multiple particle-tracking models, as might be done in a nested grid configuration.  

\end{enumerate}

\subsection{Units of Length and Time}
The GWF Model formulates the groundwater flow equation without using prescribed length and time units. Any consistent units of length and time can be used when specifying the input data for a simulation. This capability gives a certain amount of freedom to the user, but care must be exercised to avoid mixing units.  The program cannot detect the use of inconsistent units.

\subsection{Particle Mass Budget}
A summary of all inflow (sources) and outflow (sinks) of particle mass is called a mass budget.  \mf calculates a mass budget for the overall model as a check on the acceptability of the solution, and to provide a summary of the sources and sinks of mass to the flow system.  The particle mass budget is printed to the PRT Model Listing File for selected time steps.  In the current implementation, each particle is assigned unit mass, and the numerical value of the flow can be interpreted as particles per time.

\subsection{Time Stepping}
In \mf time step lengths are controlled by the user and specified in the Temporal Discretization (TDIS) input file.  When the particle-tracking model and transport model are included in the same simulation, then the length of the time step specified in TDIS is used for both models.  If the PRT Model runs in a separate simulation from the GWF Model, then ....  Instructions for specifying time steps are described in the TDIS section of this user guide; additional information on GWF and PRT configurations are in the Flow Model Interface section.  



\newpage
\subsection{PRT Model Name File}
The PRT Model Name File specifies the options and packages that are active for a PRT model.  The Name File contains two blocks: OPTIONS  and PACKAGES. The length of each line must be 299 characters or less. The lines in each block can be in any order.  Files listed in the PACKAGES block must exist when the program starts. 

Comment lines are indicated when the first character in a line is one of the valid comment characters.  Commented lines can be located anywhere in the file. Any text characters can follow the comment character. Comment lines have no effect on the simulation; their purpose is to allow users to provide documentation about a particular simulation. 

\vspace{5mm}
\subsubsection{Structure of Blocks}
\lstinputlisting[style=blockdefinition]{./mf6ivar/tex/prt-nam-options.dat}
\lstinputlisting[style=blockdefinition]{./mf6ivar/tex/prt-nam-packages.dat}

\vspace{5mm}
\subsubsection{Explanation of Variables}
\begin{description}
% DO NOT MODIFY THIS FILE DIRECTLY.  IT IS CREATED BY mf6ivar.py 

\item \textbf{Block: OPTIONS}

\begin{description}
\item \texttt{list}---is name of the listing file to create for this PRT model.  If not specified, then the name of the list file will be the basename of the PRT model name file and the '.lst' extension.  For example, if the PRT name file is called ``my.model.nam'' then the list file will be called ``my.model.lst''.

\item \texttt{PRINT\_INPUT}---keyword to indicate that the list of all model stress package information will be written to the listing file immediately after it is read.

\item \texttt{PRINT\_FLOWS}---keyword to indicate that the list of all model package flow rates will be printed to the listing file for every stress period time step in which ``BUDGET PRINT'' is specified in Output Control.  If there is no Output Control option and ``PRINT\_FLOWS'' is specified, then flow rates are printed for the last time step of each stress period.

\item \texttt{SAVE\_FLOWS}---keyword to indicate that all model package flow terms will be written to the file specified with ``BUDGET FILEOUT'' in Output Control.

\end{description}
\item \textbf{Block: PACKAGES}

\begin{description}
\item \texttt{ftype}---is the file type, which must be one of the following character values shown in table~\ref{table:ftype}. Ftype may be entered in any combination of uppercase and lowercase.

\item \texttt{fname}---is the name of the file containing the package input.  The path to the file should be included if the file is not located in the folder where the program was run.

\item \texttt{pname}---is the user-defined name for the package. PNAME is restricted to 16 characters.  No spaces are allowed in PNAME.  PNAME character values are read and stored by the program for stress packages only.  These names may be useful for labeling purposes when multiple stress packages of the same type are located within a single PRT Model.  If PNAME is specified for a stress package, then PNAME will be used in the flow budget table in the listing file; it will also be used for the text entry in the cell-by-cell budget file.  PNAME is case insensitive and is stored in all upper case letters.

\end{description}


\end{description}

\begin{table}[H]
\caption{Ftype values described in this report.  The \texttt{Pname} column indicates whether or not a package name can be provided in the name file.  The capability to provide a package name also indicates that the PRT Model can have more than one package of that Ftype}
\small
\begin{center}
\begin{tabular*}{\columnwidth}{l l l}
\hline
\hline
Ftype & Input File Description & \texttt{Pname}\\
\hline
DIS6 & Rectilinear Discretization Input File \\
DISV6 & Discretization by Vertices Input File \\
MIP6 & Model Input File \\
FMI6 & Flow Model Interface Package &  \\ 
PRP6 & Particle Release Point Package \\
OC6 & Output Control Option \\
OBS6 & Observations Option \\
\hline 
\end{tabular*}
\label{table:ftypeprt}
\end{center}
\normalsize
\end{table}

\vspace{5mm}
\subsubsection{Example Input File}
\lstinputlisting[style=inputfile]{./mf6ivar/examples/prt-nam-example.dat}



%\newpage
%\subsection{Structured Discretization (DIS) Input File}
%\input{gwf/dis}

%\newpage
%\subsection{Discretization with Vertices (DISV) Input File}
%\input{gwf/disv}

%\newpage
%\subsection{Unstructured Discretization (DISU) Input File}
%\input{gwf/disu}

\newpage
\subsection{Model Input (MIP) Package}
Model Input (MIP) Package information is read from the file that is specified by ``MIP6'' as the file type.  The MIP Package is required, and only one MIP Package can be specified for a PRT model.  The information read by the MIP Package pertains to the entire PRT model.

\vspace{5mm}
\subsubsection{Structure of Blocks}
%%\lstinputlisting[style=blockdefinition]{./mf6ivar/tex/prt-mip-options.dat}
\lstinputlisting[style=blockdefinition]{./mf6ivar/tex/prt-mip-griddata.dat}

\vspace{5mm}
\subsubsection{Explanation of Variables}
\begin{description}
% DO NOT MODIFY THIS FILE DIRECTLY.  IT IS CREATED BY mf6ivar.py 

\item \textbf{Block: OPTIONS}

\begin{description}
\item \texttt{EXPORT\_ARRAY\_ASCII}---keyword that specifies input griddata arrays should be written to layered ascii output files.

\end{description}
\item \textbf{Block: GRIDDATA}

\begin{description}
\item \texttt{porosity}---is the aquifer porosity.

\item \texttt{retfactor}---is a real value by which velocity is divided within a given cell.  RETFACTOR can be used to account for solute retardation, i.e., the apparent effect of linear sorption on the velocity of particles that track solute advection.  RETFACTOR may be assigned any real value.  A RETFACTOR value greater than 1 represents particle retardation (slowing), and a value of 1 represents no retardation.  The effect of specifying a RETFACTOR value for each cell is the same as the effect of directly multiplying the POROSITY in each cell by the proposed RETFACTOR value for each cell.  RETFACTOR allows conceptual isolation of effects such as retardation from the effect of porosity.  The default value is 1.

\item \texttt{izone}---is an integer zone number assigned to each cell.  IZONE may be positive, negative, or zero.  The current cell's zone number is recorded with each particle track datum.  If the ISTOPZONE option is set to any value other than zero in a PRP Package, particles released by that PRP Package terminate if they enter a cell whose IZONE value matches ISTOPZONE.  If ISTOPZONE is not specified or is set to zero in a PRP Package, IZONE has no effect on the termination of particles released by that PRP Package.

\end{description}


\end{description}

\vspace{5mm}
\subsubsection{Example Input File}
\lstinputlisting[style=inputfile]{./mf6ivar/examples/prt-mip-example.dat}


\newpage
\subsection{Particle Release Point Conditions (PRP) Package}
Particle Release Point (PRP) Package information is read from the file that is specified by ``PRP6'' as the file type.  More than one PRP Package can be specified for a PRT model. 

\vspace{5mm}
\subsubsection{Structure of Blocks}
\lstinputlisting[style=blockdefinition]{./mf6ivar/tex/prt-prp-options.dat}
\lstinputlisting[style=blockdefinition]{./mf6ivar/tex/prt-prp-dimensions.dat}
%%\lstinputlisting[style=blockdefinition]{./mf6ivar/tex/prt-prp-griddata.dat}
\lstinputlisting[style=blockdefinition]{./mf6ivar/tex/prt-prp-packagedata.dat}
\vspace{5mm}
\noindent \textit{FOR ANY STRESS PERIOD}
\lstinputlisting[style=blockdefinition]{./mf6ivar/tex/prt-prp-period.dat}
\packageperioddescription \: If no PERIOD block is specified for any period, a single particle is released from each release point at the beginning of the simulation.
%\noindent All of the stress period information in a PERIOD block will apply only to that stress period; the information will not continue to apply for subsequent stress periods.  Note that this behavior is different from the simple stress packages (CHD, WEL, DRN, RIV,
%GHB, RCH and EVT) and the advanced stress packages (MAW, SFR, LAK, and UZF).

\vspace{5mm}
\subsubsection{Explanation of Variables}
\begin{description}
% DO NOT MODIFY THIS FILE DIRECTLY.  IT IS CREATED BY mf6ivar.py 

\item \textbf{Block: OPTIONS}

\begin{description}
\item \texttt{BOUNDNAMES}---keyword to indicate that boundary names may be provided with the list of release-point cells.

\item \texttt{TRACK}---keyword to specify that record corresponds to track.

\item \texttt{FILEOUT}---keyword to specify that an output filename is expected next.

\item \texttt{trackfile}---name of the output file to write tracking information.

\item \texttt{TRACKCSV}---keyword to specify that record corresponds to the track CSV.

\item \texttt{trackcsvfile}---name of the comma-separated value (CSV) file to write tracking information.

\item \texttt{stoptime}---real value defining the maximum simulation time to which particles in the model can be tracked.  Particles that have not terminated earlier due to another termination condition will terminate when simulation time STOPTIME is reached.  If the last stress period in the simulation consists of more than one time step, particles will not be tracked past the ending time of the last stress period, regardless of STOPTIME.  If the last stress period in the simulation consists of a single time step, it is assumed to be a steady-state stress period, and its ending time will not limit the simulation time to which particles can be tracked.

\item \texttt{stoptraveltime}---real value defining the maximum travel time over which particles in the model can be tracked.  Particles that have not terminated earlier due to another termination condition will terminate when their travel time reaches STOPTRAVELTIME.  If the last stress period in the simulation consists of more than one time step, particles will not be tracked past the ending time of the last stress period, regardless of STOPTRAVELTIME.  If the last stress period in the simulation consists of a single time step, it is assumed to be a steady-state stress period, and its ending time will not limit the travel time over which particles can be tracked.

\item \texttt{STOP\_AT\_WEAK\_SINK}---is a text keyword to indicate that a particle is to terminate when it enters a cell that is a weak sink.  By default, particles are allowed to pass though cells that are weak sinks.

\item \texttt{istopzone}---integer value defining the stop zone number.  If cells have been assigned IZONE values in the GRIDDATA block, a particle terminates if it enters a cell whose IZONE value matches ISTOPZONE.  An ISTOPZONE value of zero indicates that there is no stop zone.  The default value is zero.

\item \texttt{DRAPE}---is a text keyword to indicate that if a particle's release point is in a cell that happens to be dry at release time, the particle is to be moved to the topmost active cell below it, if any. By default, a particle is not released into the simulation if its release point's cell is dry at release time, instead the particle is terminated immediately with ireason 3 and istatus 8.

\item \texttt{referencetime}---real value defining the time at which to release particles. This is comparable to MODPATH 7 referencetime option 1.

\end{description}
\item \textbf{Block: DIMENSIONS}

\begin{description}
\item \texttt{nreleasepts}---is the number of particle release points.

\end{description}
\item \textbf{Block: PACKAGEDATA}

\begin{description}
\item \texttt{irptno}---integer value that defines the PRP release point number associated with the specified PACKAGEDATA data on the line. IRPTNO must be greater than zero and less than or equal to NRELEASEPTS.  The program will terminate with an error if information for a PRP release point number is specified more than once.

\item \texttt{cellid}---is the cell identifier, and depends on the type of grid that is used for the simulation.  For a structured grid that uses the DIS input file, CELLID is the layer, row, and column.   For a grid that uses the DISV input file, CELLID is the layer and CELL2D number.  If the model uses the unstructured discretization (DISU) input file, CELLID is the node number for the cell.

\item \texttt{xrpt}---real value that defines the x coordinate of the release point in model coordinates.  The (x, y, z) location specified for the release point must lie within the cell that corresponds to the specified cellid.

\item \texttt{yrpt}---real value that defines the y coordinate of the release point in model coordinates.  The (x, y, z) location specified for the release point must lie within the cell that corresponds to the specified cellid.

\item \texttt{zrpt}---real value that defines the z coordinate of the release point in model coordinates.  The (x, y, z) location specified for the release point must lie within the cell that corresponds to the specified cellid.

\item \texttt{boundname}---name of the release-point cell.  BOUNDNAME is an ASCII character variable that can contain as many as 40 characters.  If BOUNDNAME contains spaces in it, then the entire name must be enclosed within single quotes.

\end{description}
\item \textbf{Block: PERIOD}

\begin{description}
\item \texttt{iper}---integer value specifying the stress period number for which the data specified in the PERIOD block apply. IPER must be less than or equal to NPER in the TDIS Package and greater than zero. The IPER value assigned to a stress period block must be greater than the IPER value assigned for the previous PERIOD block. The information specified in the PERIOD block applies only to that stress period.

\item \texttt{releasesetting}---specifies when to release particles within the stress period.  Overrides package-level REFERENCETIME option, which applies to all stress periods. By default, RELEASESETTING configures particles for release at the beginning of the specified time steps. For time-offset releases, provide a FRACTION with another RELEASESETTING option.

\begin{lstlisting}[style=blockdefinition]
ALL
FIRST
FREQUENCY <frequency>
STEPS <steps(<nstp)>
[FRACTION <fraction(<nstp)>]
\end{lstlisting}

\item \texttt{ALL}---keyword to indicate release of particles at the start of all time steps in the period.

\item \texttt{FIRST}---keyword to indicate release of particles at the start of the first time step in the period. This keyword may be used in conjunction with other keywords to release particles at the start of multiple time steps.

\item \texttt{frequency}---release particles at the specified time step frequency. This keyword may be used in conjunction with other keywords to release particles at the start of multiple time steps.

\item \texttt{steps}---release particles at the start of each step specified in STEPS. This keyword may be used in conjunction with other keywords to release particles at the start of multiple time steps.

\item \texttt{fraction}---release particles after the specified fraction of the time step has elapsed. If FRACTION is not set, particles are released at the start of the specified time step(s). FRACTION must be a single value when used with ALL, FIRST, or FREQUENCY. When used with STEPS, FRACTION may be a single value or an array of the same length as STEPS. If a single FRACTION value is provided with STEPS, the fraction applies to all steps.

\end{description}


\end{description}

\vspace{5mm}
\subsubsection{Example Input File}
\lstinputlisting[style=inputfile]{./mf6ivar/examples/prt-prp-example.dat}



\newpage
\subsection{Output Control (OC) Option}
Input to the Output Control Option of the Particle Tracking Model is read from the file that is specified as type ``OC6'' in the Name File. If no ``OC6'' file is specified, default output control is used. The Output Control Option determines how and when concentrations are printed to the listing file and/or written to a separate binary output file.  Under the default, concentration and overall transport budget are written to the Listing File at the end of every stress period. The default printout format for concentrations is 10G11.4.  The concentrations and overall transport budget are also written to the list file if the simulation terminates prematurely due to failed convergence.

Output Control data must be specified using words.  The numeric codes supported in earlier MODFLOW versions can no longer be used.

For the PRINT and SAVE options of concentration, there is no option to specify individual layers.  Whenever the concentration array is printed or saved, all layers are printed or saved.

\vspace{5mm}
\subsubsection{Structure of Blocks}
\vspace{5mm}

\noindent \textit{FOR EACH SIMULATION}
\lstinputlisting[style=blockdefinition]{./mf6ivar/tex/prt-oc-options.dat}
\vspace{5mm}
\noindent \textit{FOR ANY STRESS PERIOD}
\lstinputlisting[style=blockdefinition]{./mf6ivar/tex/prt-oc-period.dat}

\vspace{5mm}
\subsubsection{Explanation of Variables}
\begin{description}
% DO NOT MODIFY THIS FILE DIRECTLY.  IT IS CREATED BY mf6ivar.py 

\item \textbf{Block: OPTIONS}

\begin{description}
\item \texttt{BUDGET}---keyword to specify that record corresponds to the budget.

\item \texttt{FILEOUT}---keyword to specify that an output filename is expected next.

\item \texttt{budgetfile}---name of the output file to write budget information.

\item \texttt{BUDGETCSV}---keyword to specify that record corresponds to the budget CSV.

\item \texttt{budgetcsvfile}---name of the comma-separated value (CSV) output file to write budget summary information.  A budget summary record will be written to this file for each time step of the simulation.

\item \texttt{TRACK}---keyword to specify that record corresponds to a binary track file.

\item \texttt{trackfile}---name of the output file to write tracking information.

\item \texttt{TRACKCSV}---keyword to specify that record corresponds to a CSV track file.

\item \texttt{trackcsvfile}---name of the comma-separated value (CSV) file to write tracking information.

\item \texttt{TRACK\_ALL}---keyword to indicate that ...

\item \texttt{TRACK\_RELEASE}---keyword to indicate that particle tracking output is to be written when a particle is released

\item \texttt{TRACK\_TRANSIT}---keyword to indicate that particle tracking output is to be written when a particle exits a cell

\item \texttt{TRACK\_TIMESTEP}---keyword to indicate that particle tracking output is to be written at the end of each time step

\item \texttt{TRACK\_TERMINATE}---keyword to indicate that particle tracking output is to be written when a particle terminates for any reason

\item \texttt{TRACK\_WEAKSINK}---keyword to indicate that particle tracking output is to be written when a particle exits a weak sink (a cell which removes some but not all inflow from adjacent cells)

\item \texttt{TRACK\_USERTIME}---keyword to indicate that particle tracking output is to be written at user-specified times, provided as double precision values to the TRACK\_TIMES or TRACK\_TIMESFILE options

\item \texttt{TRACK\_TIMES}---keyword indicating tracking times will follow

\item \texttt{times}---times to track, relative to the beginning of the simulation.

\item \texttt{TRACK\_TIMESFILE}---keyword indicating tracking times file name will follow

\item \texttt{timesfile}---name of the tracking times file

\end{description}
\item \textbf{Block: PERIOD}

\begin{description}
\item \texttt{iper}---integer value specifying the starting stress period number for which the data specified in the PERIOD block apply.  IPER must be less than or equal to NPER in the TDIS Package and greater than zero.  The IPER value assigned to a stress period block must be greater than the IPER value assigned for the previous PERIOD block.  The information specified in the PERIOD block will continue to apply for all subsequent stress periods, unless the program encounters another PERIOD block.

\item \texttt{SAVE}---keyword to indicate that information will be saved this stress period.

\item \texttt{PRINT}---keyword to indicate that information will be printed this stress period.

\item \texttt{rtype}---type of information to save or print.  Can only be BUDGET.

\item \texttt{ocsetting}---specifies the steps for which the data will be saved.

\begin{lstlisting}[style=blockdefinition]
ALL
FIRST
LAST
FREQUENCY <frequency>
STEPS <steps(<nstp)>
\end{lstlisting}

\item \texttt{ALL}---keyword to indicate save for all time steps in period.

\item \texttt{FIRST}---keyword to indicate save for first step in period. This keyword may be used in conjunction with other keywords to print or save results for multiple time steps.

\item \texttt{LAST}---keyword to indicate save for last step in period. This keyword may be used in conjunction with other keywords to print or save results for multiple time steps.

\item \texttt{frequency}---save at the specified time step frequency. This keyword may be used in conjunction with other keywords to print or save results for multiple time steps.

\item \texttt{steps}---save for each step specified in STEPS. This keyword may be used in conjunction with other keywords to print or save results for multiple time steps.

\end{description}


\end{description}

\vspace{5mm}
\subsubsection{Example Input File}
\lstinputlisting[style=inputfile]{./mf6ivar/examples/prt-oc-example.dat}


\newpage
\subsection{Observation (OBS) Utility for a PRT Model}

PRT Model observations include the simulated groundwater concentration (\texttt{concentration}), and the mass flow, with units of mass per time, between two connected cells (\texttt{flow-ja-face}). The data required for each PRT Model observation type is defined in table~\ref{table:gwtobstype}. For \texttt{flow-ja-face} observation types, negative and positive values represent a loss from and gain to the \texttt{cellid} specified for ID, respectively.

\subsubsection{Structure of Blocks}
\vspace{5mm}

\noindent \textit{FOR EACH SIMULATION}
\lstinputlisting[style=blockdefinition]{./mf6ivar/tex/utl-obs-options.dat}
\lstinputlisting[style=blockdefinition]{./mf6ivar/tex/utl-obs-continuous.dat}

\subsubsection{Explanation of Variables}
\begin{description}
\input{./mf6ivar/tex/utl-obs-desc.tex}
\end{description}


\begin{longtable}{p{2cm} p{2.75cm} p{2cm} p{1.25cm} p{7cm}}
\caption{Available PRT model observation types} \tabularnewline

\hline
\hline
\textbf{Model} & \textbf{Observation type} & \textbf{ID} & \textbf{ID2} & \textbf{Description} \\
\hline
\endhead

\hline
\endfoot

%%%
PRT Model observations include the simulated groundwater concentration (\texttt{concentration}), and the mass flow, with units of mass per time, between two connected cells (\texttt{flow-ja-face}). The data required for each PRT Model observation type is defined in table~\ref{table:gwtobstype}. For \texttt{flow-ja-face} observation types, negative and positive values represent a loss from and gain to the \texttt{cellid} specified for ID, respectively.

\subsubsection{Structure of Blocks}
\vspace{5mm}

\noindent \textit{FOR EACH SIMULATION}
\lstinputlisting[style=blockdefinition]{./mf6ivar/tex/utl-obs-options.dat}
\lstinputlisting[style=blockdefinition]{./mf6ivar/tex/utl-obs-continuous.dat}

\subsubsection{Explanation of Variables}
\begin{description}
\input{./mf6ivar/tex/utl-obs-desc.tex}
\end{description}


\begin{longtable}{p{2cm} p{2.75cm} p{2cm} p{1.25cm} p{7cm}}
\caption{Available PRT model observation types} \tabularnewline

\hline
\hline
\textbf{Model} & \textbf{Observation type} & \textbf{ID} & \textbf{ID2} & \textbf{Description} \\
\hline
\endhead

\hline
\endfoot

%%%
PRT Model observations include the simulated groundwater concentration (\texttt{concentration}), and the mass flow, with units of mass per time, between two connected cells (\texttt{flow-ja-face}). The data required for each PRT Model observation type is defined in table~\ref{table:gwtobstype}. For \texttt{flow-ja-face} observation types, negative and positive values represent a loss from and gain to the \texttt{cellid} specified for ID, respectively.

\subsubsection{Structure of Blocks}
\vspace{5mm}

\noindent \textit{FOR EACH SIMULATION}
\lstinputlisting[style=blockdefinition]{./mf6ivar/tex/utl-obs-options.dat}
\lstinputlisting[style=blockdefinition]{./mf6ivar/tex/utl-obs-continuous.dat}

\subsubsection{Explanation of Variables}
\begin{description}
\input{./mf6ivar/tex/utl-obs-desc.tex}
\end{description}


\begin{longtable}{p{2cm} p{2.75cm} p{2cm} p{1.25cm} p{7cm}}
\caption{Available PRT model observation types} \tabularnewline

\hline
\hline
\textbf{Model} & \textbf{Observation type} & \textbf{ID} & \textbf{ID2} & \textbf{Description} \\
\hline
\endhead

\hline
\endfoot

%%%\input{../Common/prt-obs.tex}
%%%\label{table:prtobstype}
\end{longtable}

\vspace{5mm}
\subsubsection{Example Observation Input File}

An example PRT Model observation file is shown below.

\lstinputlisting[style=inputfile]{./mf6ivar/examples/utl-obs-prt-example.dat}


%%%\label{table:prtobstype}
\end{longtable}

\vspace{5mm}
\subsubsection{Example Observation Input File}

An example PRT Model observation file is shown below.

\lstinputlisting[style=inputfile]{./mf6ivar/examples/utl-obs-prt-example.dat}


%%%\label{table:prtobstype}
\end{longtable}

\vspace{5mm}
\subsubsection{Example Observation Input File}

An example PRT Model observation file is shown below.

\lstinputlisting[style=inputfile]{./mf6ivar/examples/utl-obs-prt-example.dat}



\newpage
\subsection{Flow Model Interface (FMI) Package}
Flow Model Interface (FMI) Package information is read from the file that is specified by ``FMI6'' as the file type.  The FMI Package is required, and only one FMI Package can be specified for a PRT model.

For most simulations, the PRT Model needs groundwater flows for every cell in the model grid, for all boundary conditions, and for other terms, such as the flow of water in or out of storage.  The FMI Package is the interface between the PRT Model and simulated groundwater flows provided by a corresponding GWF Model that is running concurrently within the simulation or from binary budget files that were created from a previous GWF model run.  The following are several different FMI simulation cases:

\begin{itemize}

\item Flows are provided by a corresponding GWF Model running in the same simulation---in this case, all groundwater flows are calculated by the corresponding GWF Model and provided through FMI to the transport model.  This is a common use case in which the user wants to run the flow and particle-tracking models as part of a single simulation.  The GWF and PRT models must be part of a GWF-PRT Exchange that is listed in mfsim.nam.  If a GWF-PRT Exchange is specified by the user, then the user does not need to specify an FMI Package input file for the simulation, unless an FMI option is needed.  If a GWF-PRT Exchange is specified and the FMI Package is specified, then the PACKAGEDATA block below is not read or used.

\item Flows are provided from a previous GWF model simulation---in this case FMI should be provided in the PRT name file and the head and budget files should be listed in the FMI PACKAGEDATA block.  In this case, FMI reads the simulated head and flows from these files and makes them available to the particle-trcking model.  There are some additional considerations when the heads and flows are provided from binary files.

\begin{itemize}
\item The binary budget file must contain the simulated flows for all of the packages that were included in the GWF model run.  Saving of flows can be activated for all packages by specifying ``SAVE\_FLOWS'' as an option in the GWF name file.  The GWF Output Control Package must also have ``SAVE BUGET ALL'' specified.  The easiest way to ensure that all flows and heads are saved is to use the following simple form of a GWF Output Control file:

\begin{verbatim}
BEGIN OPTIONS
  HEAD FILEOUT mymodel.hds
  BUDGET FILEOUT mymodel.bud
END OPTIONS

BEGIN PERIOD 1
  SAVE HEAD ALL
  SAVE BUDGET ALL
END PERIOD
\end{verbatim}

\item The binary budget file must have the same number of budget terms listed for each time step.  This will always be the case when the binary budget file is created by \mf.
\item The binary heads file must have heads saved for all layers in the model.  This will always be the case when the binary head file is created by \mf.  This was not always the case as previous MODFLOW versions allowed different save options for each layer.
\item If the binary budget and head files have more than one time step for a single stress period, then the budget and head information must be contained within the binary file for every time step in the simulation stress period.
\item The binary budget and head files must correspond in terms of information stored for each time step and stress period.
\item If the binary budget and head files have information provided for only the first time step of a given stress period, this information will be used for all time steps in that stress period in the PRT simulation. If the final (or only) stress period in the binary budget and head files contains data for only one time step, this information will be used for any subsequent time steps and stress periods in the PRT simulation. This makes it possible to provide flows, for example, from a steady-state GWF stress period and have those flows used for all PRT time steps in that stress period, for all remaining time steps in the PRT simulation, or for all time steps throughout the entire GWT simulation. With this option, it is possible to have smaller time steps in the PRT simulation than the time steps used in the GWF simulation. Note that this cannot be done when the GWF and PRT models are run in the same simulation, because in that case, both models are solved over the same sequence of time steps and stress periods, as listed in the TDIS Package. The option to read flows from a previous GWF simulation via Flow Model Interface may offer an efficient alternative to running both models in the same simulation, but comes at the cost of having potentially very large budget files.
\end{itemize}

\end{itemize}

\noindent Determination of which FMI use case to invoke requires careful consideration of the different advantages and disadvantages of each case.  For example, running PRT and GWF in the same simulation can often be faster because GWF flows are passed through memory to the PRT model instead of being written to files.  The disadvantage of this approach is that the same time step lengths must be used for both GWF and PRT.  Ultimately, it should be relatively straightforward to test different ways in which GWF and PRT interact and select the use case most appropriate for the particular problem. 

\vspace{5mm}
\subsubsection{Structure of Blocks}
\lstinputlisting[style=blockdefinition]{./mf6ivar/tex/prt-fmi-packagedata.dat}

\vspace{5mm}
\subsubsection{Explanation of Variables}
\begin{description}
% DO NOT MODIFY THIS FILE DIRECTLY.  IT IS CREATED BY mf6ivar.py 

\item \textbf{Block: OPTIONS}

\begin{description}
\item \texttt{SAVE\_FLOWS}---keyword to indicate that FMI flow terms will be written to the file specified with ``BUDGET FILEOUT'' in Output Control.

\item \texttt{FLOW\_IMBALANCE\_CORRECTION}---correct for an imbalance in flows by assuming that any residual flow error comes in or leaves at the concentration of the cell.  When this option is activated, the PRT Model budget written to the listing file will contain two additional entries: FLOW-ERROR and FLOW-CORRECTION.  These two entries will be equal but opposite in sign.  The FLOW-CORRECTION term is a mass flow that is added to offset the error caused by an imprecise flow balance.  If these terms are not relatively small, the flow model should be rerun with stricter convergence tolerances.

\end{description}
\item \textbf{Block: PACKAGEDATA}

\begin{description}
\item \texttt{flowtype}---is the word GWFBUDGET or GWFHEAD.  If GWFBUDGET is specified, then the corresponding file must be a budget file from a previous GWF Model run.

\item \texttt{FILEIN}---keyword to specify that an input filename is expected next.

\item \texttt{fname}---is the name of the file containing flows.  The path to the file should be included if the file is not located in the folder where the program was run.

\end{description}


\end{description}

\vspace{5mm}
\subsubsection{Example Input File}
\lstinputlisting[style=inputfile]{./mf6ivar/examples/prt-fmi-example.dat}


